% ------------------------------------------------------------------------------
% Palestra: 
% Autores:
%     Adorilson Bezerra <adorilson@gmail.com>
% Licen�a Creative Commons Atribui��o 3.0. 
% Voc� pode usar e alterar este documento, 
% mas deve obrigatoriamente citar a autoria. 
% ------------------------------------------------------------------------------

\documentclass{beamer}

% ------------------------------------------------------------------------------
\usepackage[latin1]{inputenc}
\usepackage[brazil]{babel}
\usepackage{graphicx}
\usepackage{beamerthemesplit}
\usepackage{ae}
\usepackage{alltt}
\usepackage{pslatex}
% ------------------------------------------------------------------------------

\usecolortheme{beaver}

% ------------------------------------------------------------------------------
\title[From Software Product Lines to Software Ecosystems]
{
    From Software Product Lines to Software Ecosystems
}
\subtitle{}
\author[Adorilson Bezerra]
{
    Adorilson Bezerra
}
\institute{Universidade Federal do Rio Grande do Norte\\Departamento de Inform�tica e Matem�tica Aplicada}
\date{\today}
% ------------------------------------------------------------------------------

\begin{document}

% ------------------------------------------------------------------------------
\frame{\titlepage}
% ------------------------------------------------------------------------------

\section{Apresenta��o}
    \frame
    {
        \frametitle{From Software Product Lines to Software Ecosystems}
        \begin{itemize}
            \item Jan Bosch
            \item Accepted for the 13th International Software Product Line Conference (SPLC 2009)
            \item August 24-28, 2009, San Francisco, CA
        \end{itemize}
    }
    \frame
    {
        \frametitle{Se��es}
        \begin{itemize}
            \item Introdu��o
            \item Taxonomia dos ecosistemas de software
            \item Transi��o para um ecosistema de software
            \item Implica��es para a engenharia de software
            \item Discuss�o
        \end{itemize}
    }
\section{Introdu��o}
\section{Taxonomia dos ecosistemas de software}
\section{Transi��o para um ecosistema de software}
\section{Implica��es para a engenharia de software}
\section{Discuss�o}
\end{document}
