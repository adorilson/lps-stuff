
% ==============================================================================
\chapter{Detalhamento de trabalho}

Nesse capítulo, apresentaremos a arquitetura da plataforma Android, assim como o
ciclo de vida de uma aplicação para essa plataforma e os pontos de variações 
entre os dispositivos. Por fim, discutiremos os 
mecanismos de controle das variabilidades.

% ------------------------------------------------------------------------------
\section{Plataforma Android}
Segundo \cite{whatisandroid}, o Android é uma pilha de software para dispositivos
móveis que inclui um sistema operacional, middleware e aplicações chaves. O Android
SDK fornece as ferramentas e API's necessários para o desenvolvimento de aplicações
para a plataforma, utilizando a linguagem de programação Java.

\subsection{Características}
\begin{itemize}
    \item Framework de aplicações: permite o reuso e troca de componentes.
    Os desenvolvedores têm acesso completo à mesma API que é usada pelas aplicações
    do núcleo da plataforma.
    \item Máquina virtual Dalvik: é uma máquina virtual especializada para o uso em
    dispositivos móveis. Aplicações escritas em Java são compiladas em bytecodes para
    essa máquina virtual, e não uma máquina virtual Java.
    \item Navegador integrado: navegador baseado no motor {\it open source} WebKit
    \item Otimizações gráficas: possibilitadas por uma biblioteca gráfica 2D 
    personalizada; gráficos 3D baseadas na especificação OpenGL ES 1.0 (acelaração
    por hardware opcional)
    \item SQLite: para armazenamento de dados estruturados
    \item Suporte para multimídia: suporte para os formatos mais comuns de 
    áudio, vídeo e imagem (MPEG4, H.264, MP3, AAC, AMR, JPG, PNG, GIF)
    \item Rico ambiente de desenvolvimento: incluindo um emulador de dispositivo,
    ferramentas para depuração, {\it profiling} de memória e perfomance, e um {\it plugin}
    para o Eclipe IDE
    \item Alguns recursos dependentes do dispositivo
        \begin{itemize}
            \item Telefonia GSM 
            \item Bluetooth, EDGE, 3G, e WiFi
            \item Camera, GPS, bússola, e acelerômetro
        \end{itemize}
\end{itemize}

\subsection{Arquitetura da plataforma}

A figura \ref{system-architecture} mostra os principais componentes do sistema 
operacional Android. Cada seção é descrita abaixo.

\begin{figure}[h]
    \centering
    \includegraphics[width=10cm]{img/system-architecture.jpg}
    \caption{Arquitetura da plataforma Android}
    \label{system-architecture}
\end{figure}

\subsubsection{Application}
Android será distribuido com um conjuto de aplicações centrais, incluindo um cliente
de e-mail, programa de SMS, calendário, mapas, navegador, entre outros. Todas as 
aplicações são escritas em Java.

\subsubsection{Application Framework}
Fornecendo uma plataforma de desenvolvimento aberta, Android permite aos desenvolvedores
criarem aplicações extremamente ricas e inovadores. Desenvolvedores são livres para
tirar vantagem do hardware dos dispositivos, acessar informações de localização, 
executar serviços em backgrounds, definir alarmes e notificações para a barra de 
status etc.

Desenvolvedores tem total acesso às mesmas API's do framework usadas pelas aplicações 
do núcleo. A arquitetura da aplicação é projetada para simplificar o reuso de componentes;
qualquer aplicação pode publicar suas capacidades(?) e qualquer outra aplicação pode 
fazer uso dessas capacidades (sujeito às restrições de segurança impostas pelo 
framwork). Esse mesmo mecanismo permite que componentes sejam substituídos pelo 
usuário.

Os principais componentes do Application Framework são:
\begin{itemize}
\sloppy % evita que em linhas muito longas o espaço entre as palavras seja
        % aumentado
    \item System View: uma 'view' representa um widget que aparece na tela
    \item Content Providers: permite que aplicações possam acessar dados de outras 
    aplicações, ou compartilhar seus dados com outras
    \item Resource Manager: fornece acesso para recursos que não são código, como 
    strings de localização, gráficos e arquivos de layout
    \item Notification Manager: permite que as aplicações exibam alertas customizados 
    na barra de status
    \item Activity Manager: gerencia o ciclo de vida das aplicações e fornece um 
    mecanismo comum de navegação
\fussy % dual do \sloppy
\end{itemize}

\subsubsection{Libraries}
Android inclui um conjunto de bibliotecas C/C++ usada por vários componentes do 
sistema. Esses recursos são expostos para os desenvolvedores através do "application 
framework". Algumas das principais bibliotecas são listadas a seguir:

\begin{itemize}
    \item System C library - implementação da bibliotec padrão C (libc), otimizado
    para dispositivos embarcados baseados em Linux
    \item Media Libraries - as bibliotecas suportam a maioria dos formatos mais 
    populares de audio, video e imagem, incluindo MPEG4, H.264, MP3, AAC, AMR, 
    JPG, e PNG
    \item Surface Manager - gerencia o acesso o subsistema do display e composição 
    suave de camadas 2D e 3D para múltiplos dispositivos
    \item LibWebCore - um moderno mecanismo para navegador web que possibilita tanto o 
    navegador do Android quanto um visualizador web embutido
    \item SGL - o mecanismo para gráficos 2D subjacente
    \item 3D libraries - uma implementação baseada na API OpenGL ES 1.0; as bibliotecas
    podem tanto usar a aceleração 3D por hardware (quando disponível), quanto
    a renderação 3D otimizada por software, já incluida.
    \item FreeType - renderização de fonte vetorial e bitmap
    \item SQLite - um poderoso e leve mecanismos de banco de dados relacional, 
    disponível para todas as apliações
\end{itemize}

\subsubsection{Android Runtime}

Android inclui um conjunto de bibliotecas básicas que provê a maioria das funcionalidades
disponíveis na biblioteca padrão da linguagem de programação Java.

Cada aplicação Android executa em seu próprio processo, com sua própria instância 
da máquina virtual Dalvik. Dalvik foi escrita de forma a permitir que um dispositivo
execute multiplas MV's eficientemente. A MV Dalvik executa arquivos no formato de 
Executáveis Dalvik (.dex) que é otimizado para usar o mínimo de memória possível. 
A MV é baseado em registradores, e executa classes compiladas por um compilador Java 
que transforma essas classes no formato .dex com a ferramenta "dx".

\subsubsection{Linux Kernel}

Android depende da versão 2.6 do Linux para serviços do núcleo do sistema como 
segurança, gerência de memória, gerência de processos, pilha de rede e modelo de 
drivers. O núcleo também atua como uma camada abstrata entre o hardware e o resto
da pilha de sofware.

\subsection{Visão geral sobre as aplicações Android}

Aplicações Android são escritas na linguagem de programação Java. O Android SDK
 compila o código - juntamente com seus dados e demais arquivos - em um pacote 
 Android, um arquivo com a extensão .apk. É este arquivo que será instalado nos 
 dispositivos.

Uma aplicação pode ter quatro tipos de componentes:
\begin{itemize}
    \item Activity: representa uma tela única na interface com o usuário, e podem 
    ser traduzidas para o português como atividades. As atividades
    de uma aplicação são independentes uma das outras, o que possibilita que outra aplicação 
    abra uma atividade (desde de que tenha as permissões necessárias), sem que seja
    necessário abrir a aplicação desta atividade
    Cada atividade é implementada como uma subclasse de Activity. E a definição 
    completa está em \cite{activity}
    \item Service: é um componente que executa em background, sem interface com 
    usuário. Outro componente, como uma atividade, pode iniciar o serviço e deixá-lo 
    em execução ou interagir com ele. Um serviço é implementado como uma subclasse de 
    Service. E a definição completa está em \cite{service}
    \item Content providers: gerencia um conjunto de dados compartilhados da aplicação. 
    Através deste componente, outras aplicações podem consultar ou mesmo modificar 
    dados (se for permitido). Um provedor de conteúdo é implementado como uma subclasse 
    de ContentProvider e deve implementar um conjunto padrão de API's que permite 
    às outras aplicações realizarem as transações. Um detalhamento sobre os provedores
    de conteúdo está em \cite{providers}
    \item Broadcast receivers: é um componente que responde a mensagens enviada ao 
    sistema operacional. Essas mensagens podem ser originadas do próprio sistema, 
    como um alerta de do nível de bateria, por exemplo, ou podem ser enviada por 
    aplicações. Dessa forma, podem ser úteis para fazer a integração entre aplicações.
    Um "broadcast receiver" é implementado como uma subclasse de BroadcastReceiver.
\end{itemize}

Um característica única em sistemas Android é que qualquer aplicação pode iniciar 
um componente de uma outra aplicação. Por exemplo, se você quer enviar alguma mensagem
para o Twitter, provavelmente irá utilizar uma outra aplicação que já faz isso, em vez
de desenvolver uma atividade dentro da sua própria aplicação. Você não precisará 
nem mesmo incorporar ou linkar o código do aplicação do Twitter. Em vez disso, você
simplesmente inicia a aplicação que enviará a mensagem e depois retornará para a sua 
aplicação. Para o usuário, irá parecer que o cliente do Twitter é parte da sua aplicação.

Como cada aplicação é executa em um processo separado, com permissões que restrigem 
o acesso de outras aplicações, sua aplicação não ativa diretamente um componente de 
outra aplicação. O sistema Android é que faz isso. Assim, para ativar um componente 
em outra aplicação, você deve enviar uma mensagem para o sistema que determina sua 
intenção de iniciar um componente em particular. Então, o sistema ativa o componente 
para você.

Essa flexibilidade é possível graças ao ciclo de vida das atividades, descrito na 
seção seguinte.

\subsubsection{Ciclo de vida de uma atividade}

Diferentemente de aplicações em outros sistemas, aplicações Android não possuem 
um único ponto de entrada. Não existe uma função main(), comum na classe principal 
das aplicações Java. O ciclo de vida de uma atividade é apresentada na Figura \ref{Activity_lifecycle}

\begin{figure}[h]
    \centering
    \includegraphics[width=10cm]{img/android_lecheta}
    \caption[Ciclo de vida de uma activity]{Ciclo de vida de uma activity (Fonte: \cite{lecheta})}
    \label{Activity_lifecycle}
\end{figure}

Todos os métodos apresentados na Figura \ref{Activity_lifecycle} são métodos 
da classe Activity, que deverão ser sobrescritos pelas aplicações, e a chamada a esses
métodos ficara a cargo do sistema operacional.

Além desses métodos há um outro importante método automaticamente executado e de 
grande importância do que diz respeito a salvar o estado atual da aplicação: 
{\it onSaveInstanceState(Bundle)}. Este método é executado antes da atividade entrar em 
background, permitindo que seja salvo estado dinâmico em um objeto {\it Bundle}, para depois 
ser recuperado em {\it onCreate(Bundle)}, se a atividade precisa ser recriada. No entanto,
em relação a dados persistentes é necessário que eles sejam salvos no método {\it onPause()}, 
já que o {\it onSaveInstanceState(Bundle)} não faz parte do ciclo de vida, não sendo chamado 
em todas as situações.

% ------------------------------------------------------------------------------
\section{Estudo de caso: Froid - Fortunes for Android}

Como estudo de caso, utilizamos uma aplicação disponibilizada por \cite{abp}, a 
Froid - Fortunes por Android. Essa aplicação visa disponibilizar um widget que 
mostre frases aos usuários, sendo possível buscar mais frases na Internet. 

Com base na documentação disponibilizada, definimos uma versão preliminar do feature 
model da aplicação, apresentado na Figura \ref{fortunes_fm}. Quando possível, utilizaremos
esta aplicação como exemplo para os conceitos apresentados.

\begin{figure}[h]
    \centering
    \includegraphics[width=10cm]{img/fortunes_fm}
    \caption{Feature model da aplicação Froid}
    \label{fortunes_fm}
\end{figure}

% ..............................................................................
\section{Pontos de variações em dispositivos com Android}

Nessa seção, apresentaremos os pontos de variação em dispositivos com Android,
quando possível apresentaremos um exemplo com a aplicação Froid, ou trecho diversos.

\subsection{Versão da API}

A primeira versão da API do Android foi disponibilizada em outubro de 2008. De lá 
para cá, ele vem passando por diversas modificações, sendo a maioria deles ou 
introdução ou substituições de funcionalidades. Como partes da API são autualizadas, 
as partes antigas são marcadas como obsoletas e não removidas, assim aplicações feitas
com versões antigas da API continuarão funcionando nas mais recentes.

As versões da API são identificadas por um valor inteiro único chamado de {\it API Level},
distribuidas juntamente com as versões da plataforma Android. A Figura \ref{api_level} 
apresenta a relação entre versão da plataforma e {\it API Level}, além da distribuição.

\begin{figure}[h]
    \centering
    \includegraphics[width=15cm]{img/api_level}
    \caption[Distribuição da versão da plataforma e API Level]{Distribuição da versão da plataforma e API Level (Fonte: \cite{platform_versions}) }
    \label{api_level}
\end{figure}

\subsection{Tamanhos e densidade das telas}

O Android executa em uma grande variedade de dispositivos, com diferentes tamanhos
 de telas e densidades. Para facilitar o projeto de interface para diferentes 
 configurações, a plataforma divide os atuais intervalos de tamanhos de telas e
 densidades em grupos, conforme abaixo:
\begin{itemize}
    \item Tamanhos de tela: small, normal, large, e xlarge
    \item Densidades: ldpi (low), mdpi (medium), hdpi (high), e xhdpi (extra high)
\end{itemize}
 
A Figura \ref{intervalos_telas} ilustra como os diferentes tamanhos de telas 
(em polegadas) e 
densidades (em dpi) são categorizados nesses diferentes grupos.

\begin{figure}[h]
    \centering
    \includegraphics[width=15cm]{img/intervalos_telas}
    \caption[Relação entre tamanhos de telas e densidades e seus respectivos grupos]{Relação entre tamanhos de telas e densidades e seus respectivos grupos 
        (Fonte: \cite{sup_multi_screens}) }
    \label{intervalos_telas}
\end{figure}

A Figura \ref{distribuicao_telas} apresenta a distribuição atual de tamanhos 
de tela e densidade.

\begin{figure}[h]
    \centering
    \includegraphics[width=15cm]{img/api_level}
    \caption[Distribuição atual de tamanhos de tela e densidades]{Distribuição atual de tamanhos de tela e densidades (Fonte: \cite{sup_multi_screens})}
    \label{distribuicao_telas}
\end{figure}


\subsection{Versão da OpenGL ES}

O Android oferece suporte para gráficos otimizados através de uma biblioteca 2D 
customizada. Quando suportado pelo hardware do dispositivo, gráficos 3D serão 
providos pela biblioteca OpenGL ES. A Figura \ref{distribuicao_opengl} apresenta 
a distribuição atual de versões dessa biblioteca encontradas nos aparelhos.

\begin{figure}[h]
    \centering
    \includegraphics[width=15cm]{img/opengl}
    \caption[Distribuição atual da OpenGL ES]{Distribuição atual da OpenGL ES (Fonte: \cite{opengl_versions})}
    \label{distribuicao_opengl}
\end{figure}


\subsection{Hardware e sensores do dispositivo}

Android oferece suporte para diversos tipos de sensores e hardware presentes nos 
dispositivos. Contudo, nem todos os dispositivos terão todos os tipos de sensores.
Incluem-se aqui, os mecanismos de interação do usuário com o aparelho.
Sensores e hardware que poderão ou não estar presentes nos dispositivos incluem:
\begin{itemize}
    \item Acelerômetro
    \item Câmera
    \item Sensor de luminosidade
    \item Bluetooth
    \item Wifi
    \item Live wallpaper
    \item GPS
    \item Microfone
    \item Sensor de proximidade
    \item Telefone VoIP baseado em SIP
    \item Telefonia CDMA e/ou GSM
    \item USB
    \item Mecanismos de interação
    \begin{itemize}
        \item Tela sensível a toque e/ou multitoque
        \item Teclado físico
        \item Trackball
        \item Botões de navegação (five-way navigation pad)
    \end{itemize}
\end{itemize}


\subsection{Línguas internacionais}
Os textos das mensagens no sistema operacional costumam estar no idioma nativo do usuário.
É desejável que os textos das aplicações também estejam nesse mesmo idioma, e que 
isso seja possível de forma transparente para o usuário.

% ..............................................................................
\section{Mecanismos de controle de variabilidades}

Nessa seção, discutiremos os mecanismos fornecidores na plataforma Android que 
poderão ser utilizados para tratar os diversos pontos de variações apresentados.

% ..............................................................................
\subsection{Polimorfismo}

Conforme pode ser observado na Figura \ref{fortunes_fm}, a origem das frases do 
Froid pode ser um banco de dados local ou a Internet, através de um web-service.
Essa distinção foi feita através da implementação de uma interface Java: 
{\it FortuneProvider}.
Para cada tipo de fonte, foi feita uma implementação dessa interface: 
{\it FortuneProviderDatabase} para o banco de dados e {\it FortuneProviderService} para webservice. 
Além desses, analisando o código fonte foi identificado uma terceira fonte de dados, 
{\it FortuneProviderStatic}, que contém frases definidas estaticamente no código, e é 
utilizada para testes das demais funcionalidades da aplicação.

\begin{figure}[h]
    \centering
    \includegraphics[width=10cm]{img/implementacao_FortuneProvider}
    \caption{Diagrama de classes, mostrando a interface e suas implementações}
    \label{diagrama_classes}
\end{figure}

A fim de facilitar a instanciação dos providers, foi definido uma classe 
instanciadora, {\it FortuneProviderUtils}, apresentada na Figura \ref{provider_utils}.
Na Figura \ref{instanciando_provider}, podemos ver essa classe sendo utilizada para 
instanciar um provider. Sendo que a definição da fonte, se será banco de dados, 
webservice ou estático é feita de forma manual, durante a compilação.

\begin{figure}[h]
    \centering
    \includegraphics[width=10cm]{img/FortuneProviderUtils}
    \caption{Classe FortuneProviderUtils}
    \label{provider_utils}
\end{figure}

\begin{figure}[h]
    \centering
    \includegraphics[width=10cm]{img/instaciando_provider}
    \caption{Instanciando um provider em específico}
    \label{instanciando_provider}
\end{figure}

% ------------------------------------------------------------------------------

\subsection{Compatibilidade com versões antigas da API}

A API do Android evolui oferecendo novos recursos para os desenvolvedores utilizarem.
No entanto, eles também desejam que suas aplicações continuem executando em 
dispositivos com versões anteriores, sem o novo recurso.

Isso será possível para recursos não essenciais, como um virtual mesmo que um teclado físico esteja disponível. Então você poderá escrever sua aplicação para usar o novo 
recurso sem que ela falhe em dispositivos antigos. Algumas formas de manter essa 
compatibilidade serão descritas a seguir.

\subsubsection{Pacote de compatibilidade}

O pacote de compatibilidade (Support Package) é fornecido também pelo distribuidor do 
Android SDK, e inclue "bibliotecas de compatibilidade" estática que os desenvolvedores 
poderão adicionar em suas aplicações com o fim de usar API's que não estão disponíveis 
em versões antigas da plataforma ou que oferecem API's utilitárias que não fazem 
parte da API padrão. O objetivo é simplificar o desenvolvimento oferecendo mais 
recursos para serem utilizados nas aplicações, assim não será necessário preocupações
a acerca das versões da plataforma.

\subsubsection{Reflexão}

Suponha que se deseje utilizar um novo método, introduzido em nova versão da API,
como {\it android.os.Debug.dumpHprofData(String filename)}. A classe {\it android.os.Debug}
já existia desde a primeira versão do SDK, mas o método foi adicionado somente na versão 
1.5, se você acessá-lo diretamente, sua aplicações irá falar em dispositivos antigos.

A maneira mais simples para chamar o método é através de reflexão. Será feita uma 
verificação se o método existe, e o resultado será armazenado em um objeto {\it Method}.
Quando for necessário o método ser executado, isso será feito através de 
{\it Method.invoke}. Considere a Figura \ref{class_reflect1}:

\begin{figure}[h]
    \centering
    \includegraphics[width=10cm]{img/reflect1}
    \caption[Trecho da classe Reflect onde é verificado se o método existe]{Trecho da classe Reflect onde é verificado se o método existe (Fonte \cite{back_compat})}
    \label{class_reflect1}
\end{figure}

A Figura \ref{class_reflect1} mostra que o método {\it initCompatibility} é executado
 estaticamente e é responsável por verificar se o método existe. Se sim, usa um método 
 privado com a mesma assinatura do original para poder chamá-lo. 

\begin{figure}[h]
    \centering
    \includegraphics[width=10cm]{img/reflect2}
    \caption[Trecho da classe Reflect onde o método é executado]{Trecho da classe Reflect onde o método é executado (Fonte \cite{back_compat})}
    \label{class_reflect2}
\end{figure}

A Figura \ref{class_reflect2} demostra como a aplicação pode escolher entre chamar 
a nova API ou fazer algo diferente baseado na existência ou não do novo método.

Para cada método adicional que você queira executar, deverá adicionar um atributo 
{\it Method}, um atributo inicializador, e um empacotador da chamada para a classe.

Essa abordagem ficará complexa quando o método está declarado em um classe que não 
existia anteriormente. Além disso, chamar {\it Method.invoke()} será mais lento 
que invocar o método diretamente. Esses problemas podem ser diminuídos usando uma
classe empacotadora.

% ..............................................................................
\subsubsection{Classe empacotadora}

A ideia é criar uma classe que empacota toda a nova classe. Cada método na classe 
empacotadora (wrapper) apenas executa o método real correspondente, repassando os 
parâmetros, e retorna o mesmo resultado.

Se a classe alvo e o método existem, você irá ter o mesmo comportamento que teria se 
executasse a classe diretamente. Se a classe ou método não existem, a inicialização
da classe empacotadora falha, e sua aplicação sabe que deverá evitará usar os novos 
métodos.

Suponha que a classe apresentada na Figura \ref{wrapper1} seja uma nova classe adicionada.
Nós deveremos criar uma classe empacotadora para ele, apresentada na Figura \ref{wrapper2}.

\begin{figure}[h]
    \centering
    \includegraphics[width=10cm]{img/wrapper1}
    \caption[Código da nova classe]{Nova classe (Fonte: \cite{back_compat})}
    \label{wrapper1}
\end{figure}

\begin{figure}[h]
    \centering
    \includegraphics[width=10cm]{img/wrapper2}
    \caption[Código da classe empacotadora]{Código da classe empacotadora (Fonte: \cite{back_compat})}
    \label{wrapper2}
\end{figure}

A classe empacotadora tem um método para cada construtor e cada método no original,
mais uma inicializador estático que testa se a nova classe existe. Se a nova classe
não está disponível, a inicialização de {\it WrapNewClass} irá falhar, assegurando
que classe empacotadora não seja utilizada indevidamente. O método {\it checkAvailable} 
é usado como um meio simples para forçar a inicialização da classe. A Figura \ref{wrapper3}
mostra como isso deverá ser utilizado.

\begin{figure}[h]
    \centering
    \includegraphics[width=10cm]{img/wrapper3}
    \caption[Código da aplicação utilizando a classe empacotadora]{Código da aplicação utilizando a classe empacotadora (Fonte: \cite{back_compat})}
    \label{wrapper3}
\end{figure}

Se a chamada para {\it checkAvailable} for feita com sucesso, saberemos que a nova
classe faz parte do sistema. Se falar, saberemos que ela não está presente, então
poderemos ajustar a execução da aplicações de acordo com isso.

% ..............................................................................
\subsection{Checagem da versão da API em tempo de execução}

Em alguns casos, alguns comportamentos estarão tão dependentes da versão da API, que
será necessário descobrir em tempo de execução a versão da API no dispositivo.
\cite{check_version} apresenta um caso em que o comportamento do multitoque sofreu mudanças significativas
entre as versões da API. Para resolver esse problema foi criado uma classe abstrata ({\it VersionedGestureDetector}) ,
e classes que herdavam dessa ({\it CupcakeDetector, EclairDetector e FroyoDetector}) , com implementações específicas para 3 versões da API.
Em tempo de execução, era instanciada a classe adequada, conforme mostra a Figura
\ref{fig_check_version}.

\begin{figure}[h]
    \centering
    \includegraphics[width=10cm]{img/versao}
    \caption[Instanciando objeto de acordo com a versão do Android do dispositivo]{Instanciando objeto de acordo com a versão do Android do dispositivo (Fonte: \cite{check_version})}
    \label{fig_check_version}
\end{figure}

Isso é possível, e bastante simples, graças ao atributo {\it Build.VERSION.SDK}, 
ou \\ {\it Build.VERSION.SDK\_INT} em versões mais recentes, que representa a versão 
da 
API no dispositivo, e {\it Build.VERSION\_CODES} que possui constantes com os valores 
das versões da API. Bastando fazer uma comparação entre eles para que seja determinada
a versão a ser instanciada.

% ..............................................................................
\subsection{Checagem da existência de um determinado recurso}

Devido a variação de hardware existente entre os aparelhos, alguns recursos poderão
não estar disponível. Para que as aplicações saibam se um recurso está presente 
 a API do Android disponibiliza uma classe que auxilia nessa questão: {\it PackageManager}.
Essa classe possui diversos atributos que representam recursos diversos, esses
 atributos serão utilizados como parâmetros para o método {\it PackageManager.hasSystemFeature()}, que informará se o recurso está ou não disponível. 
 
Dessa forma, as aplicações poderão utilizar o recurso, ou usar um meio alternativo.
Além disso, a documentação da API recomenda que mesmo que se saiba da existência 
do recurso, ele não deve ser acessado sem antes ser confirmado que realmente está 
disponível.

% ..............................................................................
\subsection{Atributos com modificadores final e compilação condicional}

Os mecanismos apresentados anteriormente serão úteis em tempo de execução, no entanto
algumas variações são determinados em tempo de compilação. Alguns desenvolvedores 
costumam fazer esse controle comentando e descomentado trechos de código antes de 
cada compilação, no entanto isso não é uma boa prática de programação. Torna o código
com baixa manutenabilidade e muito propenso a erros. Essa é a forma atualmente utilizada 
no Froi para determinar a fonte de dados das frases.

Há algumas formas mais elegantes de fazer esse controle, entre eles compilação condicional 
e atributos com modificadores {\it final}. Compilação condicional é uma forma 
bastante utilizada em aplicações na linguagem C, não tendo suporte nativo na linguagem 
Java, onde serão utilizados atributos com modificadores {\it final}. Um exemplo didático 
desse uso está na Figura \ref{atributo_final}

\begin{figure}[h]
    \centering
    \includegraphics[width=10cm]{img/MyClass}
    \caption{Exemplo de uso de atributo final}
    \label{atributo_final}
\end{figure}

No entanto, há pelo menos um caso para a plataforma Android que esse mecanismo não será suficiente, sendo necessário o uso da compilação condicional, que deverá ser garantida por uma 
ferrameta auxiliar. As aplicações em Android são identificadas pelo pacote que as contém,
não podendo haver duas aplicações com o mesmo pacote, ou seja o valor do modificador 
{\it package} nos arquivos Java deverá ser único. Se você deseja criar duas versões de uma aplicação, 
deverá coloca-los em pacotes diferentes. Para esse caso, será necessário a compilação 
condicional.

% ..............................................................................
\subsection{Recursos da aplicação}

A maioria dos mecanismos apresentados anteriormente são possíveis graças à linguagem Java
e seu suporte para programação orientada a objetos, não sendo exclusivos da plataforma Android.

Um dos conceitos mais importantes da plataforma Android são os Recursos da Aplicação. 
Recurso é tudo aquilo que pode ser externo ao código-fonte da aplicação, como imagens, 
strings, layouts etc. Fazendo dessa forma, esses elementos poderão ser atualizados de 
forma independente. Além disso, será possível disponibilizar recursos alternativos
que suportam configurações especificadas de dispositivos, como idiomas diferentes 
ou tamanhos de telas. Um exemplo da aplicação dessa técnica é apresentada nas Figuras 
\ref{um_layout} e \ref{dois_layout}.

\begin{figure}[h]
    \centering
    \includegraphics[width=10cm]{img/resource_devices_diagram1}
    \caption{Dois dispositivos diferentes usando o mesmo layout}
    \label{um_layout}
\end{figure}

\begin{figure}[h]
    \centering
    \includegraphics[width=10cm]{img/resource_devices_diagram2}
    \caption{Dois dispositivos diferentes, cada um usando um layout específico para o seu tamanho de tela}
    \label{dois_layout}
\end{figure}

O Froid não utiliza esse recurso para layout, só existe um layout disponível. Mas
poderia existir um para tela pequena e outra para telas grandes. Considerando as 
categorias das frases, para as telas pequenas elas poderiam ser listadas, e quando
o usuário selecionar alguma delas, abrir um outra tela com a frase. Para as telas
grandes, as frases poderiam ser exibidas na mesma tela que as categorias.

Para garantir essa compatibilidade com diferentes configurações, você deve organizar
os recursos no diretório {\it res/} do projeto, usando vários sub-diretórios que 
agruparão recursos pelo tipo e configuração. Um exemplo pode ser visto na Figuras
\ref{dir_res}.

\begin{figure}[h]
    \centering
    \includegraphics[width=5cm]{img/diretorio_resources}
    \caption{Exemplo de diretório {\it res/} e seus sub-diretórios}
    \label{dir_res}
\end{figure}

As imagens deverão ficar nos diretórios {\it drawable*}, de acordo com a densidade 
(ldpi, mdpi ou hdpi) do aparelho. Os arquivos XML de layout ficarão nos 
sub-diretórios {\it layout*}. E as mensagens do aplicativo ficarão no 
sub-diretório {\it values*}. Aqui podemos perceber que haverá um layout e um 
conjunto de mensagens padrões, e um layout e mensagens a ser utilizados pelo aparelhos com tamanho de tela "large" e com a versão 11 da API. Dentro de {\it values*} as 
mensagens serão ainda identificadas pelo idioma. 

Na plataforma Android, os recursos no diretório {\it res/} 
serão mapeados para um classe {\it R}, gerada automaticamente. Quando o desenvolvedor
precisar utilizar qualquer recurso, basta acessar os atributos dessa classe, sem 
necessidade de referenciar a configuração especifica. Isso ficará a cargo da plataforma.

Os possíveis tipos de recursos são os seguintes:
\begin{itemize}
    \item Animações
        \begin{itemize}
            \item Define animações pre-determinadas
            \item Animações interpoladas são salvas em {\it res/anim/} e acessadas 
            pela classe {\it R.anim}
            \item Animações em frame são salvas em {\it res/drawable/} e acessadas 
            pela classe {\it R.drawable}
        \end{itemize}
    \item Cores de estados
        \begin{itemize}
            \item Define recursos de cor que mudam baseadas no estada da View
            \item São salvas em {\it res/color/} e acessadas pela classe {\it R.color}
        \end{itemize}
    \item Elementos gráficos
        \begin{itemize}
            \item Define vários elementos gráficos com bitmaps ou XML
            \item São salvas em {\it res/drawable/} e acessadas pela classe {\it R.drawable}
        \end{itemize}
    \item Layout
        \begin{itemize}
            \item Define o layout das interfaces com o usuário
            \item São salvas em {\it res/layout/} e acessadas pela classe {\it R.layout}
        \end{itemize}
    \item Menus
        \begin{itemize}
            \item Define o conteúdo dos menus da aplicação
            \item São salvos em {\it res/menu/} e acessados pela classe {\it R.menu}
        \end{itemize}
    \item Strings
        \begin{itemize}
            \item Define strings, arrays de string e plurais (e inclue formataçao 
            e estilo das strings)
            \item São salvos em {\it res/values} e acessados pelas classes 
            {\it R.string, R.array e R.plurals}
        \end{itemize}
    \item Estilos
        \begin{itemize}
            \item Define a aparência e formato dos elementos da interface gráfica
            \item São salvos em {\it res/values/} e acessadas pela classe {\it R.Style}
        \end{itemize}
    \item Outros tipos
        \begin{itemize}
            \item Define valores lógicos, inteiros, dimensões, cores, e outros arrays.
            \item São salvos em {\it res/values/}, mas cada um é acessado por sub-classes
            de {\it R}, como {\it R.bool, R.integer, R.dimen} etc.
        \end{itemize}
\end{itemize}

Esses recursos são automaticamente
selecionados pela plataforma, sendo transparente para o desenvolvedor. No entanto,
em alguns caso, é necessário tomar algumas medidas, como aconteceu com o Froid.

Um dos elementos de variação mais óbvios nos dispositivos móveis é justamente o tamanho das telas. Além disso, os aparelhos podem estar na posição vertical ou retrato, e, idealmente, 
as aplicações devem se adequar corretamente, aproveitando os espaços disponíves, ou 
adequando-se ao pouco espaço. O Android já fornece um mecanismo de layout relativo,
e como a interface do Froid é bastante simples, não foi necessário a definição de 
mais de um layout, no entanto, foi necessário implementar o {\it onSaveInstanceState(Bundle)}
para que quando o usuário rotacionasse o dispositivo, a frase exibida não fosse alterada, 
conforme exibido na Figura \ref{onSaveInstanceState}. 

\begin{figure}[h]
    \centering
    \includegraphics[width=15cm]{img/onSaveInstanceState}
    \caption{Método {\it onSaveInstanceState(Bundle)} da aplicação Froid}
    \label{onSaveInstanceState}
\end{figure}

A Figura \ref{onCreate} mostra a implementação do método {\it onCreate(Bundle)} da 
atividade principal do Froid, com alguns trechos em destaque. Na região 1 veja a 
como a atividade é desenhada simplesmente chamando o método {\it setContentView()}, 
passado como padrão o nome do layout, que foi definido em um arquivo XML. Na 
região 2, vemos a verificação se há um estado saldo anterior, que, se existir, foi 
feito no método {\it onSaveInstanceState(Bundle)} e é útil quando estamos vendo uma
frase e rotacionamos o aparelho. Por fim, na região 3, temos a manipulação de outros 
recursos, como botões e outros elementos de interface.

\begin{figure}[h]
    \centering
    \includegraphics[width=15cm]{img/onCreate_anotado}
    \caption{Método {\it onCreate(Bundle)} da aplicação Froid}
    \label{onCreate}
\end{figure}

Quando o mecanismo de seleção do recurso adequado, ou quando é necessário saber as 
configurações do aparelho, pode-se utilizar o método {\it getResources()} e carregar 
o recurso adequado de acordo com algum critério. No caso da Figura \ref{multiple_screens},
um determinado layout é selecionado quando a menor dimensão do aparelho é maior que 600.

\begin{figure}[h]
    \centering
    \includegraphics[width=10cm]{img/multiple_screens}
    \caption{Carregando um recurso de acordo com uma configuração do aparelho}
    \label{multiple_screens}
\end{figure}


% ..............................................................................
\subsection{Compartilhamento de informações}

Um recurso bastante útil no Android é a fácil integração entre aplicações através 
do {\it Content Providers}.
Isso permite que aplicações como o Froid possam enviar mensagens para redes sociais,
por exemplo, sem escrever o cliente para tais redes. A Figura \ref{share} mostra 
como o Froid cria uma mensagem e a envia para o sistema. Os aplicações que podem 
responder a essa mensagem serão listados, incluindo aí clientes de Facebook, Twitter e aplicativo de SMS, e o usuário deverá escolher qual delas 
deseja usar.

\begin{figure}[h]
    \centering
    \includegraphics[width=10cm]{img/share}
    \caption{Código do Froid responsável pelo compartilhamento de frases}
    \label{share}
\end{figure}

% ..............................................................................
\subsection{Android NDK}

A arquitetura do Android foi desenvolvida de forma a permitir que os aplicações de  desenvolvedores, por intermédio do Application Framework, não se preocupem com os 
hardwares específico dos dispositivos. As especificidades de cada hardware ficará a cargo 
do componentes da Libraries.No entanto, em alguns casos poderá ser necessário desenvolvimento em código nativo. Para tal, é disponibilizado o Android NDK. 

O Android NDK é um conjunto de ferramentas que permite embarcar componentes de código 
nativo nas aplicações Android. Esses componentes poderão ser escritos em C e C++.
A última versão do NDK suporta esses conjuntos de instruções ARM:

\begin{itemize}
    \item ARMv5TE (incluindo instruções Thumb-1)
    \item ARMv7 (incluindo Thumb-2 e instruções VFPv3-D16)
    \item instruções x86
\end{itemize}

O NDK fornece cabeçalhos para libc (a biblioteca C), libm, (a biblioteca Math), 
OpenGL ES, a interface JNI, entre outras bibliotecas.

