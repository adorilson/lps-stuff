\chapter{Introdução}

\section{Motivação}
O mercado de dispositivos móveis tem crescido muito rapidamente nos últimos anos.
Previsões apontam que [NÚMERO DE DISPOSITIVOS NOS PRÓXIMOS ANOS....]. Desenvolvedores
de aplicações desejam disponibilizar seus aplicativos para o máximo número de 
dispositivos. Dispositivos estes que possuem diversas diferenças entre si, 
principalmente considerando smartphones e tablets: tamanho e qualidade de tela, 
existência ou não de recursos como telefone GSM, bluetooth, EDGE, 3G, WiFi, câmera, 
GPS, bússola, e acelerômetro entre outros.

Um dos fatores que influenciam na escolha de um aparelho é o sistema operacional.
O Android é um dos sistemas operacionais mais utilizados no mundo, estando 
disponível em mais de [NÚMERO DE APARELHOS COM ANDROID] tipos de aparelhos. 
O que reforça a necessidade de mecanismos para gerenciar as diversas variações entre 
os aparelhos.

Além disso, irão existir também variações de regras de negócios e recursos da aplicação.
Desenvolvedores irão desejar facilidades para gerenciar diferentes produtos mas que 
compartilham um núcleo comum.

Linha de produtos de software é uma família de software que compartilham um núcleo 
comum. Os mecanismos providos pelas plataformas de desenvolvimento para para o gerência 
de variabilidades entres diferentes produtos é um importante aspectos a ser considerado
e compreendido pelos desenvolvedores.

% ------------------------------------------------------------------------------
%\section{Motivação}
Dessa forma, é necessário determinarmos os possíveis pontos 
de variação da plataforma Android, assim como entender os mecanismos que a plataforma 
oferece para auxiliar o gerenciamento das variabilidades.

Com isso, seremos capazes de criar aplicações que possam ser executadas em dispositivos
com características distintas, além de gerenciar variabilidades determinadas pelas 
regras de negócios.

% ------------------------------------------------------------------------------
%\section{Problema}
%Manter uma aplicação que seja executada em dispositivos com caracteristicas distintas \\
%Gerenciar variabilidades determinadas pelas regras de negocios das aplicações

\section{Limitação dos trabalhos propostos}
Executando consultas nos mecanismos de buscas na Internet, assim como em diretórios 
de trabalhos acadêmicos e eventos, praticamente não encontra-se trabalhos relacionados
a linha de produtos de software e Android. Os poucos trabalhos retornados apenas citam
o Android, mas o foco são outras plataformas. Aqueles que tratam do Android, discutem 
aspectos que não o controle de variabilidades.

Um único trabalho que supostamente trata do uso de linha de produtos de software
para o desenvolvimento de uma família de aplicações para Android está em \cite{gnios}.
No entanto, não foi possível acesso ao código fonte da LPS, tampouco àlguma publicação
relacionada. 

Em contato por e-mail, nos foi informado que este projeto foi parte de uma
projeto de mestrado e que, infelizmente, o código fonte havia sido perdido. Foi 
acrescentado que o Android SDK não gerencia variabilidades, sendo a aplicação distribuída 
na forma de uma arquivo APK, que pode ser instalado em qualquer dispositivo. E, 
considerando que o hardware da plataforma é fragmentado, o conceito de variabilidades
de produtos é útil aqui. Por fim, foi acrescentado que o família de produtos Travel
Excel foi projetada com a FeatureHouse IDE, disponível em \cite{featurehouse}.

\section{Contribuição do trabalho}
O presente trabalho enumera os mecanismos de controle de variabilidades que podem
ser utilizados em aplicações desenvolvidos para a plataforma Android. Para tanto, 
a metodologia utilizada foi o estudo da plataforma, através da documentação e 
blogs oficiais, comunidades diversas de desenvolvedores e o estudo de trabalhos 
relacionados, ainda que escassos em se tratando da tecnologia Android, leitura de 
código-fonte de projetos de código aberto e um estudo de caso.

\section{Estrutura do trabalho}
A proxima seção faz isso e aquilo, a proxima aquilo outro... O trabalho é concluído 
na ultima seção.

