\chapter{Conclusão}

A plataforma Android oferece diversos mecanismos para o tratamento de variabilidades
em aplicações. No entanto, esses mecanismos vão no caminho de permitir a criação 
de um único APK capaz de ser instalado em qualquer dispositivo. Do ponto de vista 
de mercado isso é bom, na medida que o usuário não precisará pensar sobre qual 
versão deveria instalar no seu dispositivo, caso outra abordagem fosse escolhida.

Por outro lado, os pacotes poderão ficar demasiadamente e desnecessariamente grande.
Carregarão recursos que nunca serão utilizados por um dado dispositivo, e também 
terão trechos de código que nunca serão executados.

Além disso, existem as variabilidades própria da aplicação. Atualmente, um modelo 
de distribuição bastante comum é existência de diversas versões da aplicações, com 
mais ou menos recursos, por exemplo, pode-se ter uma versão gratuita e uma versão paga. 
Embora a compilação condicional, aliada a outras técnicas possam ser utilizadas para essa distinção, é necessário aprofundar os estudos nessa área.
